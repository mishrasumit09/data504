\newenvironment{transitionframe}{
  \setbeamercolor{background canvas}{bg=yellow}
  \begin{frame}}{
    \end{frame}
}


\setbeamercolor{frametitle}{fg=blue}
\setbeamercolor{title}{fg=black}
\setbeamertemplate{footline}[frame number]
\setbeamertemplate{navigation symbols}{} 
\setbeamertemplate{itemize items}{-}
\setbeamercolor{itemize item}{fg=blue}
\setbeamercolor{itemize subitem}{fg=blue}
\setbeamercolor{enumerate item}{fg=blue}
\setbeamercolor{enumerate subitem}{fg=blue}
\setbeamercolor{button}{bg=MyBackground,fg=blue,}


% If you like road maps, rather than having clutter at the top, have a roadmap show up at the end of each section 
% (and after your introduction)
% Uncomment this is if you want the roadmap!
% \AtBeginSection[]
% {
%    \begin{frame}
%        \frametitle{Roadmap of Talk}
%        \tableofcontents[currentsection]
%    \end{frame}
% }
\setbeamercolor{section in toc}{fg=blue}
\setbeamercolor{subsection in toc}{fg=red}
\setbeamersize{text margin left=1em,text margin right=1em} 

\newenvironment{wideitemize}{\itemize\addtolength{\itemsep}{10pt}}{\enditemize}


\AtBeginSection[]{
  \begin{frame}
  \vfill
  \centering
  \begin{beamercolorbox}[sep=8pt,center,shadow=true,rounded=true]{title}
    \usebeamerfont{title}\insertsectionhead\par%
  \end{beamercolorbox}
  \vfill
  \end{frame}
}

%\setbeamertemplate{caption}[numbered]
\usepackage{adjustbox}
\usepackage{threeparttable}
\usepackage{graphicx}
\usepackage{multicol}
\usepackage{natbib}
\usepackage{float}
\usepackage{url}
\usepackage{hyperref}
\usepackage{color}
\usepackage{tabls, calc}
\usepackage[tableposition=top]{caption}
\usepackage{ifthen}
\usepackage{setspace}
\usepackage{booktabs}
\usepackage[english]{babel}
\usepackage[latin1]{inputenc}
\usepackage{caption}
\usepackage{subfigure}	
%\usepackage{custom}
\usepackage{tikz}
\usepackage{array}
\usepackage{dcolumn}
\usepackage{multirow}
\usepackage{markdown}
\usepackage{siunitx}
\usepackage{tfrupee}
%\usepackage[table]{xcolor}
%\usepackage{xcolor}
\usepackage{colortbl}
\usepackage{tcolorbox}

\tcbset{width=0.9\textwidth,boxrule=0pt,colback=red,arc=0pt,auto outer arc,left=0pt,right=0pt,boxsep=5pt}


%%%%%%%%%%%%%%%%%%%%%%%%%%%%%5
\usepackage{listings}
\definecolor{codegreen}{rgb}{0,0.6,0}
\definecolor{codegray}{rgb}{0.5,0.5,0.5}
\definecolor{codepurple}{rgb}{0.58,0,0.82}
\definecolor{backcolour}{rgb}{0.95,0.95,0.92}

\lstdefinestyle{mystyle}{
    backgroundcolor=\color{backcolour},   
    commentstyle=\color{codegreen},
    keywordstyle=\color{magenta},
    numberstyle=\tiny\color{codegray},
    stringstyle=\color{codepurple},
    basicstyle=\ttfamily\footnotesize,
    breakatwhitespace=false,         
    breaklines=true,                 
    captionpos=b,                    
    keepspaces=true,                 
    numbers=left,                    
    numbersep=5pt,                  
    showspaces=false,                
    showstringspaces=false,
    showtabs=false,                  
    tabsize=2
}

\lstset{style=mystyle}


%%%%%%%%%%%%%%%%%%%%%%%%%%%%
\newcommand{\Note}[1]{
\rule{2.5cm}{0.25pt} \\ \textit{\footnotesize{\textcolor{rubineRed}{Note:} \textcolor{darkerGray}{#1}}}}

\newcommand{\formula}[1]{
\begin{beamerboxesrounded}[shadow = false, lower = formula body]{}
#1
\end{beamerboxesrounded}
}

\setbeamercolor{prac ques body}{fg=oiB}

\newcommand{\pq}[1]{
\begin{beamerboxesrounded}[shadow = false, lower = prac ques body]{}
#1
\end{beamerboxesrounded}
}



%%%%%%%%%%%%%%%%%%%%%%%%%%%%%%%
\newcommand{\hl}[1]{\textit{\textcolor{hlblue}{#1}}}
\newcommand{\hlGr}[1]{\textit{\textcolor{lightGreen}{#1}}}
\newcommand{\mathhl}[1]{\textcolor{hlblue}{\ensuremath{#1}}}


%%%Define Colors Here
\definecolor{applegreen}{rgb}{0.55, 0.71, 0.0}
\definecolor{ao(english)}{rgb}{0.0, 0.5, 0.0}
\definecolor{pink}{rgb}{0.858, 0.188, 0.478}
\definecolor{purple}{HTML}{6A5ACD}
\definecolor{orange}{HTML}{FFA500}
\definecolor{denim}{rgb}{0.08, 0.38, 0.74}
\definecolor{seagreen}{rgb}{0.13, 0.7, 0.67}
\definecolor{tangerine}{rgb}{0.95, 0.52, 0.0}
\definecolor{textboxgreen}{rgb}{0.67, 0.88, 0.69}
\definecolor{textboxred}{HTML}{BF616A}
\definecolor{redPink}{HTML}{e64173}
\definecolor{turquoise}{HTML}{20B2AA}
\definecolor{backcolour}{rgb}{0.95,0.95,0.92}
\xdefinecolor{oiB}{rgb}{0.22,0.52,0.72}
\definecolor{oiG}{rgb}{.298,.447,.114}
\xdefinecolor{hlblue}{rgb}{0.051,0.65,1}
\xdefinecolor{gray}{rgb}{0.5, 0.5, 0.5}
\xdefinecolor{darkGray}{rgb}{0.3, 0.3, 0.3}
\xdefinecolor{darkerGray}{rgb}{0.2, 0.2, 0.2}
\xdefinecolor{rubineRed}{rgb}{0.89,0,0.30}
\xdefinecolor{irishGreen}{rgb}{0,0.60,0}	
\definecolor{lightGreen}{rgb}{0.387,0.581,0.148} 
\setbeamercolor{frametitle}{fg=denim}

%$%%%%%%%%%%%%%%%%%%%%%%%%%%%%%%%%%
\newcommand{\dq}[1]{
\begin{beamerboxesrounded}[shadow = false, lower = disc ques body]{}
#1
\end{beamerboxesrounded}
}

% solnMult: solutions for practice questions

\newcommand{\solnMult}[1]{
\item[] \vspace{-0.59cm}
\only<1>{\item #1}
\soln{\only<2->{\item \orange{#1}}}
}

% removepagenumbers
\newcommand{\removepagenumbers}{% 
  \setbeamertemplate{footline}{}
}

\newcommand{\soln}[1]{\textit{#1}}
\newcommand{\solnGr}[1]{#1}


%%%%%%%%%%%%%%%%
\usepackage{geometry}
\usepackage{graphicx}
\usepackage{amssymb}
%\usepackage{cancel}
\usepackage{epstopdf}
\usepackage{amsmath}  	% this permits text in eqnarray among other benefits
\usepackage{url}		% produces hyperlinks
\usepackage{hyperref}	% allows for color usage in tables
\usepackage[english]{babel}
\usepackage[latin1]{inputenc}
\usepackage{colortbl}	% allows for color usage in tables
\usepackage{multirow}	% allows for rows that span multiple rows in tables
\usepackage{color}		% this package has a variety of color options
\usepackage{pgf}
\usepackage{calc}
\usepackage{ulem}
\usepackage{multicol}
\usepackage{textcomp}
%\usepackage{txfonts}
\usepackage{listings}
\usepackage{tikz}
\usepackage{array}
\usepackage{wasysym}
\usepackage{fancyvrb}
%%%%%%%%%%%%%%%%

%%%%%%%%%%%%%%%%
% Custom commands
%%%%%%%%%%%%%%%%

% degree
\newcommand{\degree}{\ensuremath{^\circ}}


% cite
\newcommand{\ct}[1]{
\vfill
{\tiny #1}}


% Remember
\newcommand{\Remember}[1]{\textit{\scriptsize{\textcolor{orange}{Remember:} #1}}}

% expected counts
\newcommand{\ex}[1]{\textit{\textcolor{blue}{#1}}}

% red
\newcommand{\red}[1]{\textit{\textcolor{rubineRed}{#1}}}

% pink
\newcommand{\pink}[1]{\textit{\textcolor{rubineRed!90!white!50}{#1}}}

% green
\newcommand{\green}[1]{\textit{\textcolor{irishGreen}{#1}}}

% orange
\newcommand{\orange}[1]{\textit{\textcolor{orange}{#1}}}

% links: webURL, webLin, appLink
\newcommand{\webURL}[1]{\urlstyle{same}{ \textit{\textcolor{darkGray}{\url{#1}}}}}
\newcommand{\webLink}[2]{\href{#1}{\textcolor{darkGray}{{#2}}}}
\newcommand{\appLink}[2]{\href{#1}{\textcolor{white}{{#2}}}}

% mail
\newcommand{\mail}[1]{\href{mailto:#1}{\textit{\textcolor{darkGray}{#1}}}}


% two col: two columns
\newenvironment{twocol}[4]{
\begin{columns}[c]
\column{#1\textwidth}
#3
\column{#2\textwidth}
#4
\end{columns}
}




% slot (for probability calculations)
\newenvironment{slot}[2]{
\begin{array}{c} 
\underline{#1} \\ 
#2
\end{array}
}

% pr: left and right parentheses
\newcommand{\pr}[1]{
\left( #1 \right)
}


% cancel
\newcommand{\cancel}[1]{%
    \tikz[baseline=(tocancel.base)]{
        \node[inner sep=0pt,outer sep=0pt] (tocancel) {#1};
        \draw[red, line width=0.5mm] (tocancel.south west) -- (tocancel.north east);
    }%
}


%%%%%%%%%%%%%%%%
% Custom boxes
%%%%%%%%%%%%%%%%

% app: application exercise

\setbeamercolor{app body}{fg=oiG}

\newcommand{\app}[1]{
\begin{beamerboxesrounded}[shadow = false, lower = app body]{}
#1
\end{beamerboxesrounded}
}


%%%%%%%%%%%%%%%%
% Change margin
%%%%%%%%%%%%%%%%

\newenvironment{changemargin}[2]{%
\begin{list}{}{%
\setlength{\topsep}{0pt}%
\setlength{\leftmargin}{#1}%
\setlength{\rightmargin}{#2}%
\setlength{\listparindent}{\parindent}%
\setlength{\itemindent}{\parindent}%
\setlength{\parsep}{\parskip}%
}%
\item}{\end{list}}

%%%%%%%%%%%%%%%%
% Footnote
%%%%%%%%%%%%%%%%

\long\def\symbolfootnote[#1]#2{\begingroup%
\def\thefootnote{\fnsymbol{footnote}}\footnote[#1]{#2}\endgroup}

%%%%%%%%%%%%%%%%
% Commands from the book
%%%%%%%%%%%%%%%%

\newenvironment{data}[1]{\texttt{#1}}{}
\newenvironment{var}[1]{\texttt{#1}}{}
\newenvironment{resp}[1]{\texttt{#1}}{}

%%%%%%%%%%%%%%%%
% Graphics
%%%%%%%%%%%%%%%%

\DeclareGraphicsRule{.tif}{png}{.png}{`convert #1 `dirname #1`/`basename #1 .tif`.png}
